$\sim$$\sim$$\sim$ R\+A\+D\+\_\+\+R\+E\+AD R\+E\+A\+D\+ME $\sim$$\sim$$\sim$

//////////////////////////////////// // // // // // P\+S\+QL -\/ A basic tutorial // // // // // ////////////////////////////////////

Step 1\+: P\+S\+QL install

If on current linux, we can find postgre\+S\+QL on synaptic package manager. Simply search under \char`\"{}name\char`\"{} for \char`\"{}postgresql\char`\"{}. Everything should work out nicely, no dpkg required.

Step 2\+: Creating a database

Open a shell (I like bash). We\textquotesingle{}re gonna change user to root, then change to user \char`\"{}postgres\char`\"{} from there. To accomplish this, we do\+:

sudo su

su -\/ postgres

Now, we can create psql users and databases. For now, we\textquotesingle{}ll stick with user \char`\"{}postgres\char`\"{}. To create a database, we do\+:

createdb -\/O postgres testdb

This creates a database called \char`\"{}testdb\char`\"{} with owner \char`\"{}postgres\char`\"{}.

Now, before we go any further, we have to take a look at the psql network settings. Find the \char`\"{}pg\+\_\+hba.\+conf\char`\"{} file that was installed with psql. By default, it should be located at /etc/postgresql/$<$version-\/number$>$/main/. Open this file, and take a look at the line under the heading \char`\"{}\+I\+Pv4 local
  connections\char`\"{}. Make a note of the IP address, then close the file. This is the local address we will use to make a connection to a local psql database. Usually, this address will be 127.\+0.\+0.\+1.

Step 3\+: Getting into the database

We can now make alterations to our database. But before that, we have to set a password for our new user \char`\"{}postgres\char`\"{}. We can do this in another shell (with the default user). Open a shell, and do\+:

sudo -\/u postgres psql postgres

This will provide access into the psql shell layer. From there, we can do

postgres

By default, no password is set. Set the password to whatever you want, just don\textquotesingle{}t forget it!

Now that we\textquotesingle{}ve done that, go back to the shell in which you have changed to user \char`\"{}postgres\char`\"{}. To get into the database we created above, we can do\+:

psql postgres -\/h $<$I\+P-\/address$>$ -\/d testdb

This will prompt for a password, which you should have now. We operating within the database and can make edits.

Step 4\+: Adding data

Tables are used to hold data within psql databases. To create a table while operating within a table, we can do\+:

create table $<$table-\/name$>$ ( $<$data-\/spec1$>$ $<$data-\/type$>$ $<$data-\/spec2$>$ $<$data-\/type$>$ ... ... )

In this instantiation, the data-\/spec refers to the name of the data column. some examples of this could be \char`\"{}names\char`\"{}, \char`\"{}phonenumbers\char`\"{} or \char`\"{}addresses\char`\"{}. The data-\/type signifies what form the variables within the data columns will take. Examples are \char`\"{}integer\char`\"{}, varchar(20) (which indicates a string of 20 chars or less) and char(10) (which indicates a numeral string of 10 digits or less). 